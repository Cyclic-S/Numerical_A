\section{第十五周数值分析实验}
\subsection{迭代法}
\begin{ex}
设线性方程组 $\left\{\begin{array}{l}8 x_1-3 x_2+2 x_3=20, \\ 4 x_1+11 x_2-x_3=33, \\ 6 x_1+3 x_2+12 x_3=36 .\end{array}\right.$ 精确解是 $x^*=(3,2,1)^T$. 按如下格式求解
$$
\left\{\begin{array}{l}
	x_1{ }^{(k+1)}=\left(3 x_2{ }^{(k)}-2 x_3{ }^{(k)}+20\right) / 8 \\
	x_2{ }^{(k+1)}=\left(-4 x_1{ }^{(k)}+x_3{ }^{(k)}+33\right) / 11 \\
	x_3{ }^{(k+1)}=\left(-6 x_1{ }^{(k)}-3 x_2{ }^{(k)}+36\right) / 12
\end{array}\right. .
$$

其中 $x^{(0)}=(0,0,0)^T$. 试给出第 $10,20,100,200,1000$ 次的迭代结果及误差向量的无穷范数.
\end{ex}
\lstinputlisting[language=matlab]{w15/q1.m}
\qa 
% Table generated by Excel2LaTeX from sheet 'Sheet1'
\begin{table}[H]
	\centering
	\caption{运行结果}
	\begin{tabular}{c|ccccc}
		& 10    & 20    & 100   & 200   & 1000 \\
		\hline
		$x_1$    & 3.0000 &3.0000& 3     & 3     & 3 \\
		$x_2$    &1.9999& 2.0000& 2     & 2     & 2 \\
		$x_3$    & 0.9999 & 1.0000 & 1     & 1     & 1 \\
		误差向量的无穷范数 &1.26E-04 & 3.84E-09 & 0     & 0     & 0 \\
	\end{tabular}%
	\label{tab:addlabelw15-1}%
\end{table}%

\subsection{高斯塞德尔迭代法}
\begin{ex}
	用高斯一赛德尔迭代法求解练习 1 , 试给出第 $10,20,100,200,1000$ 次的迭代结果及误差向量的无穷范数.
\end{ex}
\lstinputlisting[language=matlab]{w15/q2.m}
\qa 
% Table generated by Excel2LaTeX from sheet 'Sheet1'
\begin{table}[H]
	\centering
	\caption{运行结果}
	\begin{tabular}{c|ccccc}
		& 10    & 20    & 100   & 200   & 1000 \\
		\hline
		$x_1$    & 3.0000 & 3     & 3     & 3     & 3 \\
		$x_2$    & 2.0000 & 2     & 2     & 2     & 2 \\
		$x_3$    & 1.0000 & 1     & 1     & 1     & 1 \\
		误差向量的无穷范数 & 6.32E-09 & 0     & 0     & 0     & 0 \\
	\end{tabular}%
	\label{tab:addlabel15-2}%
\end{table}%

