\section{第十三周数值分析实验}
\subsection{矩阵的LU分解}
\begin{ex}
	实现n阶非奇异矩阵A的LU分解程序,函数格式为:
	$[L,U]=lu\_decomposition(A)$
	并计算P153, 例5.
\end{ex}
\lstinputlisting[language=matlab]{w13/lu_decomposition.m}
\lstinputlisting[language=matlab]{w13/q1.m}
\qa 
\begin{eqnarray*}
	\begin{bmatrix}
		1 & 2 & 3\\
		2 & 5 & 2\\
		3 & 1 & 5
	\end{bmatrix}
	=\begin{bmatrix}
		1&0&0\\
		2&1&0\\
		3&-5&1
	\end{bmatrix}
	\begin{bmatrix}
		1&2&3\\
		0&1&-4\\
		0&0&-24
	\end{bmatrix}
	\Longrightarrow
	\boldsymbol{x}=
	\begin{bmatrix}
		1\\2\\3
	\end{bmatrix}.
\end{eqnarray*}

\subsection{矩阵的Cholesky分解}
\begin{ex}
	实现n阶对称正定矩阵A的Cholesky分解程序,函数格式为:
	
	$L=cholesky\_Factorization (A)$
	并计算P177第9题. (MATLAB自带程序为chol(A)).
\end{ex}
\lstinputlisting[language=matlab]{w13/cholesky_Factorization.m}
\lstinputlisting[language=matlab]{w13/q2.m}
\qa 
\scriptsize{
	\begin{align*}
		\begin{array}{lc}
			\begin{bmatrix}
				1.41421356237310&0&0&0&0\\
				-0.707106781186548&1.22474487139159&0&0&0\\
				0&-0.816496580927726&1.15470053837925&0&0\\
				0&0&-0.866025403784439&1.11803398874990&0\\
				0&0&0&-0.894427190999916&1.09544511501033
			\end{bmatrix}
		\end{array}
	\end{align*}
}
$\boldsymbol{x}=\left[0.833333333333333,0.666666666666667,0.500000000000000,0.333333333333333,0.166666666666667\right]^T$